% !TEX root = index.tex


\section{Appendix: Fundamental Forms}

Now for some hardcore differential geometry. We learned to compute curvatures for curvatures in the special case when the surface is the graph of a function and the point is a critical point. This is not a practical way to compute curvatures and we need a more ``coordinate independent'' way of doing this. This is achieved using the so called \textbf{fundamental forms}.

\textbf{Curves:} We did a similar thing for curves, we first found the curvature as acceleration if the speed is constant and then discovered that ...





First we start with parametrizing surfaces. We'll describe a surface $ S$ as the image of a function
\begin{align*}
  \Phi: \R^2_{u,v} \rightarrow \R^3
\end{align*}
The domain can be an open subset of $ \R^2$. For example,

{\color{red} \textbf{ examples of parametrizations }}.

If you unravel the definition of a derivative then you'll see that the vectors
\begin{align}
  \Phi_{u}, \Phi_{v}
\end{align}
are tangent to the surface $ S$. We can find the \textbf{unit normal vector} to the surface geometrically or using the formula
\begin{align}
  \vec{\mathbf{n}} = \dfrac{\Phi_{u} \times \Phi_{v}}{||\Phi_{u} \times \Phi_{v}||}
\end{align}

Now we can define the fundamental forms:






\subsection{First Fundamental Form}
  The first fundamental form is defined as
  \begin{align}
    \mathbb{F} :=
    \begin{bmatrix}
      \Phi_u \cdot \Phi_u & \Phi_u \cdot \Phi_v \\
      \Phi_u \cdot \Phi_v & \Phi_v \cdot \Phi_v
    \end{bmatrix}signed
  \end{align}
  If you know inner products, this is the inner product {\color{red} \textbf{ ... }}







  \subsection{Second Fundamental Form}
  The second fundamental form is defined as:
  \begin{align}
    \mathbb{S} :=
    \begin{bmatrix}
      \Phi_{uu} \cdot \vec{\mathbf{n}} & \Phi_{uv} \cdot \vec{\mathbf{n}} \\
      \Phi_{uv} \cdot \vec{\mathbf{n}} & \Phi_{vv} \cdot \vec{\mathbf{n}}
    \end{bmatrix}
  \end{align}

  The curvatures are now given by
  \begin{align}
    H &= \tr{\mathbb{S} \mathbb{F}^{-1}}/2 \\
    \kappa &= \det{\mathbb{S} \mathbb{F}^{-1}}
  \end{align}

Unfortunately, the principal directions are harder to find using this definition, but the principal curvatures themselves are easily found from the Mean and Gaussian curvatures.

\subsection{Gaussian Curvature using Fundamental Forms}









\iffalse
\section{Examples}

\subsection{Sphere}
    \begin{align}
      x^2 + y^2 + z^2 &= r^2
    \end{align}

  Parametrization:
  \begin{align}
    z &= r \cos \phi\\
    x &= r \cos \theta \sin \phi\\
    y &= r \sin \theta \sin \phi
  \end{align}

  First Derivatives:
  \begin{align}
    \Phi_{\theta} =
    \begin{bmatrix}
      0 \\
      -r \sin \theta \sin \phi \\
      r \cos \theta \sin \phi
    \end{bmatrix}
    &&
    \Phi_{\phi} =
    \begin{bmatrix}
      -r \sin \phi \\
      r \cos \theta \cos \phi \\
      r \sin \theta \cos \phi
    \end{bmatrix}
  \end{align}

  First Fundamental Form:
  \begin{align}
    \Phi_{\theta}.\Phi_{\theta} &= r^2 \sin^2 \phi \\
    \Phi_{\theta}.\Phi_{\phi} &= 0 \\
    \Phi_{\phi}.\Phi_{\phi} &= r^2
  \end{align}

  Normal vector:
  \begin{align}
    \vec n = \begin{bmatrix} \cos \phi\\
    \cos \theta \sin \phi\\
    \sin \theta \sin \phi \end{bmatrix}
  \end{align}

  Second Derivatives:
  \begin{align}
    \Phi_{\theta \theta} =
    \begin{bmatrix}
      0 \\
      -r \cos \theta \sin \phi \\
      -r \sin \theta \sin \phi
    \end{bmatrix}
    &&
    \Phi_{\phi \theta} =
    \begin{bmatrix}
      0 \\
      -r \sin \theta \cos \phi \\
      r \cos \theta \cos \phi
    \end{bmatrix}
    &&
    \Phi_{\phi \phi} =
    \begin{bmatrix}
      -r \cos \phi \\
      -r \cos \theta \sin \phi \\
      -r \sin \theta \sin \phi
    \end{bmatrix}
  \end{align}

  Second Fundamental form:
  \begin{align}
    \Phi_{\theta \theta}.\vec n &= -r \cos^2 \phi \\
    \Phi_{\theta \phi}.\vec n &= 0 \\
    \Phi_{\phi \phi}.\vec n &= -r
  \end{align}
\fi
