% !TEX root = index.tex

\setcounter{section}{-1}
\section{Introduction}
\epigraph{There is nothing more deceptive than an obvious fact.}{Sherlock Holmes}

We have an intuitive understanding of what it means for something to be curved. A circle has some curvature where as a line has no curvature. The \emph{larger} the radius of the circle the \emph{smaller} the curvature (after all a line is a circle with infinite radius). Similarly for surfaces, a sphere is curved but a plane is not. The larger the radius of the sphere the smaller the curvature (a plane is a sphere with infinite radius).

\begin{center}
\begin{tabular}{|l|l|}
  \hline Line & No curvature \\
   Circle & Curvature $\sim?$ 1/radius \\
   Sine Curve & Varying Curvature \\
   Plane & No curvature \\
   Sphere & Curvature $\sim?$ 1/radius \\
   Cylinder & ??\\
  \hline
\end{tabular}
\end{center}
But there is a subtle difference between curves and surfaces. It is possible to take a string and form a circle without stretching or compressing it, but it is not possible to take a flat piece of paper and mold it into a sphere without stretching or compressing.\footnote{Neglect the thickness.}

Or is it?

Hmmm.\\

The curvature of the sphere is in some sense \emph{intrinsic} to the sphere whereas the curvature of a circle isn't. This difference between curves and surfaces was first quantified by Gauss using what's now called Gaussian Curvature and later generalized by Riemann leading to the creation of the field of Riemannian Geometry.\\

In this class, we'll learn how linear algebra and calculus naturally help us figure out the \emph{correct} definition of Curvature for surfaces - Riemann's generalization of this to higher dimensions lies at the heart of Riemannian Geometry. We'll try to understand the geometric significance of the various kinds of curvature - Gaussian, Mean, and Principal, the statement of Gauss' Theorema Egregium, and what it means for the Gaussian Curvature to be \emph{intrinsic} to the surface.
